\section{Introduction}  \label{sec:intro}


This document provides the  science validation approach for the  LSST  .   Validation   determines whether the system 
as delivered, meets the needs of the scientific community and can deliver the scientific promise of  LSST , as outlined in 
the  LSST   science requirements document \citeds{LPM-17}. 

These papers give an overview of the science potential of Gaia  Data Release   2. Without going into in-depth analysis, they give a short introduction to selected science topics that can be addressed with this release. The purpose of these papers is to demonstrate the scientific quality of the data through examples. These performance verification papers are not exhaustive or definitive treatments of the topics.

Over a long period this could diverge significantly and choices made years ago may not be so valid anymore. So we should continually
challenge requirements and ensure they are still valid and that we are interpreting them correctly

Evolution of technologies 

Validation checks  whether the specifications in the requirements document capture the science goals of the  SRD  

Who performs Science  Validation   testing?

It is not feasible to engage the entire  LSST   scientific user community in validation testing
\begin{itemize}
\item The  DM  Science team product owners who act as proxies for the users
\item Commissioning   team scientists 
\item selected representatives of  LSST   Science collaborations 
\end{itemize}

% Poor communication or understanding of requirements 
Why do we need to do  Validation   testing?
\begin{itemize}
\item Requirements may have not been fully captured for the science goals of the system 
\item Interpretation of the requirements may have resulted in the inclusion of undesired features 
\item Requirements may not have been correctly communicated to, or understood by developers. 
\item Writing requirements in 2014 for a system in 2022 is hard - they change 
\end{itemize}

% from pstn-033

Writing concise and testable requirements is very difficult. Writing requirements in 2005 for
a system to run in 2022 is extremely difficult but is the case for  DM  . Assumptions are made
about requirements and how to implement them, but the perspective of the requirement
writer and implementer are usually not identical. Over a long period this could diverge significantly and choices made years ago may not be so valid anymore. So we should continually
challenge requirements and ensure they are still valid and that we are interpreting them correctly.

\subsection{Objectives}   label{ssec:objectives} 

Explain the integrated all-system approach 

The main purpose of science validation is to validate the end-to-end flow of the integrated system . It is black-box like where testers use teh system as is via public interfaces 

\subsection{Scope}   label{ssec:scope}

\subsection{Assumptions}   label{ssec:assumptions }


\subsection{team}   label{ssec:team}


\subsection{Entry Criteria}   label{ssec:criteria}

\begin{itemize}
\item Released version of code available 
\item All unit tests pass and system and integration testing completed
\end{itemize}

