\documentclass[DM,lsstdraft,toc]{lsstdoc}
% lsstdoc documentation: https://lsst-texmf.lsst.io/lsstdoc.html
\input{meta}

% Package imports go here.

% Local commands go here.

%If you want glossaries
%\input{aglossary.tex}
%\makeglossaries

\title{LSST Science Validation Plan}

% Optional subtitle
% \setDocSubtitle{A subtitle}

\author{%
Leanne Guy, Keith Bechtol, Jeff Carlin, Colin Slater,....and other contributors
}	

\setDocRef{LSE-439}
\setDocUpstreamLocation{\url{https://github.com/lsst/lse-439}}

\date{\vcsDate}

% Optional: name of the document's curator
\setDocCurator{Leanne Guy}
\setDocAbstract{%
This is the LSST Science ValidationPlan. We define the goals, scope and timelines for LSST Science Validation. 
}

% Change history defined here.
% Order: oldest first.
% Fields: VERSION, DATE, DESCRIPTION, OWNER NAME.
% See LPM-51 for version number policy.`
\setDocChangeRecord{%
  \addtohist{1}{2019-06-27}{First Draft}{Leanne Guy}
}

\begin{document}

% Create the title page.
\maketitle
% Frequently for a technote we do not want a title page  uncomment this to remove the title page and changelog.
% use \mkshorttitle to remove the extra pages

% ADD CONTENT HERE
% You can also use the \input command to include several content files.
\section{Introduction}  \label{sec:intro}

This document provides the Science Validation Plan for the Rubin Observatory Legacy Survey of Space and Time (LSST). 
This Science Validation Plan describes the plans for validating the data products produced by the Rubin Observatory LSST Science Pipelines. 

\subsection{Objectives}   label{ssec:objectives} 

We describe the approach to scientific validation of the LSST data products together with the various constraints and limitations. 
We present the validation program and timeline for the partially and fully integrated Rubin Observatory system. 


\subsection{Scope}   label{ssec:scope}


\subsection{Assumptions}   label{ssec:assumptions }




\section{Approach} \label{sec:approach}

This document describes the approach adopted for scientifically validating the  LSST data products and services. 

The  LSST Science Requirements  are described in \citeds{LPM-17}.


LSST Science Validation is a global validation of all the LSST Data Products and Services. Traversal across all data products and services.  it employs an 'external view' in that the process looks at the final data products from the eyes of an end science users. Validation is not done on data products stored in intermediate data formats such as FITS files, nor is validation carried out using independent scripts or command line bespoke tools or unpublished fields,  but rather using the LSST Science Platform , the scalable , etc ... LSST end user interface to the LSST archive, using only and all the published data. 

LSST science  validation is in a sense the last line of defence to find and either correct or document issues, features with the LSST data release before going public. There will typically only be a short period of time during which to run science validation, a few months before delivery 

\subsection{Proposed high-level schedule}

Map in a proposed schedule with the pre-operations DPs and how we would use this to ramp up 


Examples of validation activities involving the community (reference the LDM-503 document and the goal to include the community) that might be planned out as hack weeks
photo-z Hack Week .  -- testing the various algorithms for photo-z to include values in teh catalogue .
Define a hack week vs a sprint.


Examples of validation activities involving the community (reference the LDM-503 document and the goal to include the community) that might be planned out as sprints 
Several very specific goals, e.g validation of the astrometry, uncertainties and correlations, Catalogue completeness, etc. Themed with a validation goal. 

\section{Motivation}

What is the motivation for this work. The motivation is to provide a independent scientific validation of the final data products and services with a view of  the science that they can enable, and not to whether they meet requirements as outlined in the SRD. This activity will provide a global view and validation of the integrated system, building on top of the individual  component  - and to feedback. resolve if strange features are found

\subsection{}
Assessment of the statistical properties of the catalogue, goal is to provide guidelines to the community in using the data, avoid misinterpretations. 

\subsection{Documentation}
Validation results will contribute to the documentation of a Data Release/Preview


\subsection{LSST Science Guidelines}
This section should cover what contributors, commissioning, community can do with the early access they get via a Project sanctioned activity, e.g publications, sharing. It will oobviously go hand-in-hand with the data rights policy

\subsection{Pass/fail criteria}
\label{sec:passfail}


\section{Methods Tools, environment}

Validation should be done, where possible using an Automated integrated environment, ad-hoc exploration and data requests, notebooks, 

\section{Feedback}

Do/can we modify, reprocess and make an updated release? What is the process? 

Do we delay the release to reprocess?

Do we omit certain results and defer to a later release ? What would be the criteria to decide this? How bad does a problem need to be

Do we release and add the known issues to the documentation?

\section{Validation Organization}

\subsection{Science theme}
e.g Astrometry, photometry, variables, solar system, etc 	


\subsection{Methods}
Internal: Using LSST data only

External catalogues: Using crossmatch catalogues for comparison, independent measurements can provide an independent validation, where possible (depth, pointing, etc)

Can  models help to explain observed features to determine whether they are features or artefacts?

Statistical methods / multidimensional analysis 

	

\appendix
% Include all the relevant bib files.
% https://lsst-texmf.lsst.io/lsstdoc.html#bibliographies
\section{References} \label{sec:bib}
\renewcommand{\refname}{} % Suppress default Bibliography section
\bibliography{local,lsst,lsst-dm,refs_ads,refs,books}

% Make sure lsst-texmf/bin/generateAcronyms.py is in your path
\section{Acronyms} \label{sec:acronyms}
\addtocounter{table}{-1}
\begin{longtable}{p{0.145\textwidth}p{0.8\textwidth}}\hline
\textbf{Acronym} & \textbf{Description}  \\\hline

DM & Data Management \\\hline
FITS & Flexible Image Transport System \\\hline
LDM & LSST Data Management (document handle) \\\hline
LPM & LSST Project Management (Document Handle) \\\hline
LSE & LSST Systems Engineering (Document Handle) \\\hline
LSST & Legacy Survey of Space and Time (formerly Large Synoptic Survey Telescope) \\\hline
SRD & LSST Science Requirements; LPM-17 \\\hline
\end{longtable}

% If you want glossary uncomment below -- comment out the two lines above
%\printglossaries

\end{document}

